\chapter{Введение}

\noindent{\small\itshape Понятие исчисления. Первые примеры -- три исчисления для логики высказываний (аксиоматическое, натуральное и секвенциальное). Логический подход к синтаксису естественного языка -- исчисление Ламбека. Две формулировки ассоциативного исчисления Ламбека (натуральное и секвенциальное). Как конструктивное доказательство разрешимости исчисления позволяет построить алгоритм поиска доказательств. Варианты исчисления Ламбека.}

\section{Исчисления для логики высказываний}

Чтобы задать исчисление, нужно определить множество рассматриваемых логических выражений (т.е. формальный язык), множество аксиом, правила вывода и определить понятие вывода, используемое в системе. 

Первое исчисление, которое мы рассмотрим, было разработано Д.~Гильбертом в начале XX-го века и представляло собой формализацию аксиоматических рассуждений, используемых в математике. В такой системе некоторые из выражений, принадлежащих формальному языку, принимаются за исходные (аксиомы). Правило вывода всего одно (\textit{modus ponens}). Задача построения вывода состоит в том, чтобы предъявить способ, которым некоторое выражение формального языка может быть получено с помощью правил вывода на основе аксиом. Получаемое в результате текстовое описание проделанных в ходе этого процесса операций называется \textit{выводом} или \textit{доказательством}. Доказательство должно удовлетворять некоторым критериям корректности, поэтому определение понятия (корректного) вывода -- обязательная часть формулировки системы. 

В качестве первого примера рассмотрим исчисление гильбертовского типа для логики высказываний. Язык логики высказываний содержит выражения вида $P$, $\neg P$, $P \vee Q$, $P \wedge Q$, $P \to Q$, где $P$ и $Q$ ...

Для определения множества аксиом используются т.н. схемы аксиом. Каждая схема задаёт множество выражений, получающихся путем подстановки выражений языка логики высказываний вместо букв $A$, $B$ и $C$.

\begin{itemize}
    \item[] $A_1$: $A \to (B \to A)$
    \item[] $A_2$: $(A \to (B \to C)) \to ((A \to B) \to (A \to C))$
    \item[] $A_3$: $(A \wedge B) \to A$
    \item[] $A_4$: $(A \wedge B) \to B$
    \item[] $A_5$: $A \to (B \to (A \wedge B))$
    \item[] $A_6$: $A \to (A \vee B)$
    \item[] $A_7$: $B \to (A \vee B)$
    \item[] $A_8$: $(A \to C) \to ((B \to C) \to ((A \vee B) \to C))$
    \item[] $A_9$: $(A \to B) \to ((A \to \neg B) \to \neg A)$
    \item[] $A_{10}$: $A \to (\neg A \to B)$
\end{itemize}
 
Единственное правило вывода выглядит следующим образом: $A, A \to B \vdash B$.

Понятие вывода, применяемое в данной системе -- т.н. \textit{линейный вывод}. Пусть $\Gamma$ -- произвольное конечное множество формул. Вывод из множества формул $\Gamma$ в исчислении $C$ -- это конечная последовательность формул $F_1, \dots, F_n$ такая, что $F_i$ -- либо аксиома, либо формула из $\Gamma$, либо является следствием каких-либо двух формул, предшествующих $F_i$ в этом списке, по правилу \textit{modus ponens}. Если последняя формула в списке -- $F$, то весь этот список называется \textit{выводом} $F$. Если вывод $F$ из $\Gamma$ существует, говорят, что $F$ \textit{выводима из} $\Gamma$ и пишут $\Gamma \vdash F$. Если $\Gamma = \varnothing$ и $F$ выводима, пишут $\vdash F$.

Доказательство в исчислении гильбертовского типа мы будем изображать с помощью таблицы, в которой в первом столбце указываются номера строк для дальнейших ссылок, во втором столбце указывается выражение формального языка, включаемое в вывод, в третьем столбце указывается либо обозначение схемы аксиом, в соответствии с которой было получено выражение в данной строке, либо, если выражение получено по правилу modus ponens, указываются номера использованных для его получения выражений.

\begin{example}
Доказать $\vdash A \to A$.

\[
\Jproof{
    \proofline{A \to ((A \to A) \to A)}{A_1}
    \proofline{A \to ((A \to A) \to A)) \to ((A \to (A \to A)) \to (A \to A))}{A_2}
    \proofline{(A \to (A \to A)) \to (A \to A)}{\textit{modus ponens} 1,2}
    \proofline{A \to (A \to A)}{A_1}
    \proofline{A \to A}{\textit{modus ponens} 3,4}
}
\]
\end{example}

Идея формализации процедуры синтаксического анализа естественного языка в виде логического исчисления предлагалась неоднократно. Обыкновенно различают два направления, опирающиеся на эту идею. Первое направление связано с именами Айдукевича, Бар-Хиллела, Ламбека. Этот подход начинает развиваться с 1930-х годов, он по сути чисто логический, авторы разрабатывают большое число модификаций специализированного исчисления, т.н. исчисления Ламбека, предложенного в работе \parencite{lambek1958mathematics}, исследуют их свойства. Исчислению Ламбека будут посвящены дальнейшие разделы. Другой подход -- более поздний (80-е годы), связанный с разработкой логических языков программирования и специализированных компьютерных систем для обработки текстов, получивший в англоязычной литературе наименование ``parsing as deduction'' (термин введен в \parencite{pereira1983parsing}). Основная цель данного наравление -- изучение свойств алгоритмов синтаксического анализа, таких как разрешимость, вычислительная сложность, неоднозначность, корректность и полнота, существование нормальной формы и эффективной процедуры нормализации и пр. В качестве еще одной иллюстрации вывода в системе гильбертовского типа приведем формализацию алгоритма Кока-Янгера-Касами для контекстно-свободных языков в соответствии с подходом ``parsing as deduction'' (по \parencite{kallmeyer2010parsing}).

Чтобы применить аналогичную технику к синтаксическому анализу выражений контекстно-свободных языков, потребуются некоторые уточнения. Наша цель -- формализовать процедуру синтаксического анализа с помощью алгоритма Кока-Янгера-Касами в виде линейного вывода. Данный алгоритм требует, чтобы контекстно-свободная грамматика была в нормальной форме Хомского, т.е. продукционные правила могут быть всего двух видов: $A \to BC$ и $A \to a$, где $A, B, C$ -- нетерминалы, $a$ -- терминал.
Формальный язык будет включать выражения вида $[A, \; i, \; j]$, где $A$ -- нетерминал, $i$ и $j$ -- индексы начала и конца подстроки в строке. 

Правила вывода:

\begin{itemize}
    \item[scan]
        \begin{prooftree}
          \AxiomC{ }
          \RightLabel{{\small $A \to w_i \in P$}}
          \UnaryInfC{$[A, i-1, i]$}
        \end{prooftree}
    \item[complete]
        \begin{prooftree}
          \AxiomC{$[B, i, j]$}
          \AxiomC{$[C, j, k]$}
          \RightLabel{{\small $A \to BC \in P$}}
          \BinaryInfC{$[A, i, k]$}
        \end{prooftree}
\end{itemize}

Выражение справа от черты здесь понимается как условие, которое должны быть выполнено, чтобы применение данного правила стало возможным. Точнее, для правила \textit{scan} условие состоит в том, что в грамматике имеется продукционное правило $A \to w_i$, где $w_i$ -- $i$-ая словоформа в анализируемой строке. Аналогично, условие для правила \textit{complete} -- в грамматике имеется продукционное правило $A \to BC$. 

Целью анализа является построение вывода (т.е. доказательство) терма $[S, \; 0, \; n]$, где $S$ -- начальный символ грамматики.

\begin{example}
КС грамматика $G = \langle N, T, P, S \rangle$, где $N = \{S, NP, VP, PP, D, P, N, V\}$ -- множество нетерминалов, $T = \{man, Mary, saw, the\}$ -- множество терминалов (словоформ), $S$ -- начальный символ, множество продукционных правил $P$ содержит следующие правила:  
\begin{itemize}
    \item[] $S \to NP \; VP$
    \item[] $NP \to D \; N$
    \item[] $VP \to V \; NP$
    \item[] $N \to man$
    \item[] $NP \to Mary$
    \item[] $V \to saw$
    \item[] $D \to the$
\end{itemize}

Анализируемое предложение: \textit{Mary saw the man}.

Структурное описание:

\Tree [.S [.NP Mary ] [.VP [.V saw ] [.NP [.D the ] [.N man ] ] ] ]

Синтаксический разбор как дедукция в системе гильбертовского типа:

\[
\Jproof{
    \proofline{[NP, \; 0, \;1]}{scan (Mary)}
    \proofline{[V, \; 1, \; 2]}{scan (saw)}
    \proofline{[D, \; 2, \; 3]}{scan (the)}
    \proofline{[N, \; 3, \; 4]}{scan (man)}
    \proofline{[NP, \; 2, \; 4]}{complete 3,4}
    \proofline{[VP, \; 1, \; 4]}{complete 2,5}
    \proofline{[S, \; 0, \; 4]}{complete 1,6}
}
\]

Тот же разбор в традицонном виде (\textit{CYK chart}):

\begin{tabular}[t]{ccccc}
\cline{2-2}
\multicolumn{1}{c|}{3} & \multicolumn{1}{c|}{{[}S, 0, 4{]}}  &                                     &                                     &                                    \\ \cline{2-3}
\multicolumn{1}{c|}{2} & \multicolumn{1}{c|}{}               & \multicolumn{1}{c|}{{[}VP, 1, 4{]}} &                                     &                                    \\ \cline{2-4}
\multicolumn{1}{c|}{1} & \multicolumn{1}{c|}{}               & \multicolumn{1}{c|}{}               & \multicolumn{1}{c|}{{[}NP, 2, 4{]}} &                                    \\ \cline{2-5} 
\multicolumn{1}{c|}{0} & \multicolumn{1}{c|}{{[}NP, 0, 1{]}} & \multicolumn{1}{c|}{{[}V, 1, 2{]}}  & \multicolumn{1}{c|}{{[}D, 2, 3{]}}  & \multicolumn{1}{c|}{{[}N, 3, 4{]}} \\ \cline{2-5} 
                       & Mary                                & saw                                 & the                                 & man                               
\end{tabular}

\end{example}


\section{Исчисление Ламбека как секвенциальная система}

Предложена в работе \parencite{lambek1958mathematics}, см. тж. \parencite{moot2012logic}.

Секвенция -- основной элемент вывода в исчислении секвенций -- это формальное выражение вида $A_1, \dots, A_n \vdash B_1, \dots, B_m$, где $A_i$ и $B_j$ -- выражения формального языка (напр., языка логики высказываний). Последовательность слева от знака $\vdash$ (``штопора'') называется \textit{антецедентом}, последовательность справа -- \textit{сукцедентом}. Последовательность символов $A, B$ и т.~д. может сокращенно обозначаться символом $\Gamma$ или $\Delta$, возможно, с индексом. Вывод в исчислении секвенций представляет собой дерево, в котором над и под горизонтальной чертой записываются различные секвенции, справа от черты указывается примененное правило. Чтобы такое дерево представляло собой корректный вывод, позиции листьев в дереве должны занимать секвенции вида $A \vdash A$, т.~н. \textit{основные секвенции}. Правила вывода в секвенциальном исчислении Ламбека следующие:

\begin{prooftree}
  \AxiomC{$\Gamma, B, \Gamma' \vdash C$}
  \AxiomC{$\Delta \vdash A$}
  \RightLabel{{\small $\textbackslash_h$}}
  \BinaryInfC{$\Gamma, A, A \textbackslash B, \Gamma' \vdash C$}
\end{prooftree}

\begin{prooftree}
  \AxiomC{$A, \Gamma \vdash C$}
  \RightLabel{{\small $\textbackslash_i, \Gamma \neq \epsilon$}}
  \UnaryInfC{$\Gamma \vdash A \textbackslash C$}
\end{prooftree}

\begin{prooftree}
  \AxiomC{$\Gamma, B, \Gamma' \vdash C$}
  \AxiomC{$\Delta \vdash A$}
  \RightLabel{{\small $/_h$}}
  \BinaryInfC{$\Gamma, B/A, \Delta, \Gamma' \vdash C$}
\end{prooftree}

\begin{prooftree}
  \AxiomC{$\Gamma, A \vdash C$}
  \RightLabel{{\small $/_i, \Gamma \neq \epsilon$}}
  \UnaryInfC{$\Gamma \vdash C/A$}
\end{prooftree}

\begin{prooftree}
  \AxiomC{$\Gamma, A, B, \Gamma' \vdash C$}
  \RightLabel{{\small $\bullet_h$}}
  \UnaryInfC{$\Gamma, A \bullet B, \Gamma' \vdash C$}
\end{prooftree}

\begin{prooftree}
  \AxiomC{$\Delta \vdash A$}
  \AxiomC{$\Gamma \vdash B$}
  \RightLabel{{\small $\bullet_i$}}
  \BinaryInfC{$\Delta, \Gamma \vdash A \bullet B$}
\end{prooftree}

\begin{prooftree}
  \AxiomC{$\Gamma \vdash A$}
  \AxiomC{$\Delta_1, A, \Delta_2 \vdash B$}
  \RightLabel{{\small \textit{сечение}}}
  \BinaryInfC{$\Delta_1, \Gamma, \Delta_2 \vdash B$}
\end{prooftree}

\begin{prooftree}
  \AxiomC{}
  \RightLabel{{\small \textit{аксиома}}}
  \UnaryInfC{$A \vdash A$}
\end{prooftree}

Замечание: \parencite{moot2012logic} пишут $B \textbackslash A$ там, где (Steedman et al.) пишут $A \textbackslash B$.

Неформально, символы $A, B, C, \Gamma, \Delta$ представляют собой синтаксические категории, приписанные различным подстрокам анализируемой строки. Цель построения вывода -- показать, каким образом на основе информации о принадлежности отдельных словоформ тем или иным синтаксическим категориям, содержащейся в словаре, возможно приписать синтаксическую категорию строке, составленной из этих словоформ.

\begin{example} Предложение \textit{John loves Mary}, лексикон: $John \vdash np$, $Mary \vdash np$, $loves \vdash (np \textbackslash s)/np$, требуется построить вывод для секвенции $np, \, (np \textbackslash s)/np, \, np \vdash s$.

\begin{prooftree}
  \AxiomC{$s \vdash s$}
  \AxiomC{$np \vdash np$}
  \RightLabel{{\small $\textbackslash_h$}}
      \BinaryInfC{$np, \, np \textbackslash s \vdash s$}
      \AxiomC{$np \vdash np$}
      \RightLabel{{\small $/_h$}}      
      \BinaryInfC{$np, \, (np \textbackslash s)/np, \, np \vdash s$}
\end{prooftree}
\end{example}

\begin{example} Вывод правила подъема типа
\begin{prooftree}
  \AxiomC{$B \vdash B$}
  \AxiomC{$A \vdash A$}
  \RightLabel{{\small $/_h$}}
      \BinaryInfC{$B/A, \, A \vdash B$}
      \RightLabel{{\small $\textbackslash_i$}}      
      \UnaryInfC{$A \vdash (B/A) \textbackslash B$}
\end{prooftree}
\end{example}

\section{Исчисление Ламбека как система натурального вывода}

В работе \parencite{konig1989parsing} рассматривается система натуральной дедукции для исчисления Ламбека без произведений.

Строгая нормализация и пр. Алгоритм поиска доказательств для исчисления Ламбека без произведений в системе натурального вывода в стиле Генцена.

\section{Расширения исчисления Ламбека}

Неассоциативное исчисление Ламбека было введено в работе \parencite{lambek1961calculus}.

Субструктурные логики. Неассоциативное исчисление Ламбека как фрагмент линейной логики Жирара. Сети доказательств. Интерактивный поиск доказательств в системе \textit{Grail3}.
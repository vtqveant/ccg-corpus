\chapter{Синтаксическая разметка для <<Открытого корпуса>>}

\section{Система интерактивного поиска доказательств для исчисления Ламбека}

Мне известны следующие реализации данного подхода 
\begin{itemize}
	\item система Grail3 \parencite{moot2002proof} 
	\item библиотека Icharate для Coq \parencite{anoun2007approche}
	\item CatLog -- парсер и прувер на языке Prolog в рамках Type-Logical Grammar, \parencite{morrill2012catlog}
	\item серия работ по Natural Language Inference (NLI) в Coq в рамках теоретико-типового подхода к семантике, \parencite{chatzikyriakidis2014natural}
\end{itemize}

\section{Синтаксически размеченный корпус на основе <<Открытого корпуса>>}

О CCG и слабо контекстно-зависимых языках. О Хокенмайер и CCGBank \parencite{capelletti2009parsing}.


Об <<Открытом корпусе>>. О синтаксически размеченных корпусах для русского языка (СинТагРус).


Обоснование выбора формализма.

\begin{itemize}
    \item Категориальные грамматики, грамматики зависимостей, грамматики составляющих в духе GB. 
    \item Сравнение CCG и логического направления в категориальных грамматиках. 
    \item Слабо контекстно-зависимые языки и русский язык.
    \item Возможные применения. Категориальные грамматики и синтактико-семантический интерфейс. 
\end{itemize}


Использование категориальных грамматик в качестве базового синтаксического формализма позволит построить статистически обученный синтактико-семантический анализатор, допускающий как логические семантические представления, так и дистрибутивные семантические репрезентации. Категориальные грамматики могут быть очень выразительными (мы хотим быть в классе слабо контекстно-зависимых языков) и при этом допускают естественные механизмы контроля выразительности (мультимодальные расширения). Подход к разметке на основе интерактивного поиска доказательств позволяет реализовать разметку корпуса в приемлемые сроки и при этом обеспечить его высокое качество (применение подхода Хокенмейер для русского языка затруднительно, т.к. на русском языке нет общедоступного размеченного корпуса, который можно было бы конвертировать в нужный нам формализм.)
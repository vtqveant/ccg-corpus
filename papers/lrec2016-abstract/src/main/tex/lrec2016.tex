\documentclass[a4paper]{article}

\usepackage{afterpage}

\usepackage{authblk}
\renewcommand\Affilfont{\itshape\small}

%%% Кодировки и шрифты %%%
\usepackage[T2A]{fontenc}
\usepackage[utf8]{inputenc}
\usepackage[russian,english]{babel}
\usepackage{amssymb,amsfonts,amsmath,amsthm,amscd,mathtext}

\usepackage{cmap}   % чтобы работал поиск по pdf
\usepackage{graphicx}
\pdfcompresslevel=9

%% красная строка
\usepackage{indentfirst}
\setlength{\parindent}{5.5ex}   % 5 символов

%% интерлиньяж
\linespread{1.1}                   

%%% Общее форматирование
\usepackage[singlelinecheck=off,center]{caption} % Многострочные подписи
\usepackage{soul} % Поддержка переносоустойчивых подчёркиваний и зачёркиваний

%%% Гиперссылки %%%
\usepackage[plainpages=false,pdfpagelabels=false]{hyperref}

\usepackage{geometry}
\geometry{a4paper,top=2cm,bottom=2cm,left=2cm,right=1cm}

%%% Выравнивание и переносы %%%
\sloppy
\clubpenalty=10000
\widowpenalty=10000

%%% biblatex
\usepackage{csquotes} % recommended by pdflatex
\usepackage[%
    style = gost-authoryear-min,
    singletitle = false,
    autolang = other,
    language = auto,
    backend = biber,
    defernumbers = true,
    sortlocale = ru
]{biblatex}
\addbibresource{bibliography/bibliography.bib}

\defbibenvironment{bibliography}
{\enumerate{}
{\setlength{\leftmargin}{\bibhang}%
\setlength{\itemindent}{-\leftmargin}%
\setlength{\itemsep}{\bibitemsep}%
\setlength{\parsep}{\bibparsep}}}
{\endenumerate}
{\item}


\usepackage{hyperref}
\hypersetup{%
    colorlinks = true,
    hidelinks = false
}

%% graphics
\graphicspath{{images/}}

%% math environments
\newtheoremstyle{example-style}% name
{5pt}  % space above 
{5pt}  % space below 
{}     % body font
{\parindent}  % indent 
{\bfseries}  % theorem head font
{.}  % punctuation after theorem head
{.5em}  % space after theorem head 
{}  % theorem head spec (can be left empty, meaning 'normal')

\theoremstyle{example-style}
\newtheorem{example}{Example}


%% Nat­u­ral de­duc­tion proofs in styles used by Jaśkowski and Kal­ish and Mon­tague
\usepackage{natded}

%% Gentzen style natural deduction proof trees
\usepackage{bussproofs}
\usepackage{latexsym}

%%% linguistic packages
\usepackage{tikz-qtree,tikz-qtree-compat}   % regular trees (e.g. GB style)
\usepackage{tikz-dependency}                % dependency trees (bracket style)
\usetikzlibrary{matrix,arrows}              % for commutative diagrams

\usepackage{gb4e}  % numbered lists for linguistic examples (IMPORTANT: If you use gb4e package, let it be the last \usepackage call in the document's preamble. Otherwise you may get exceeded parameter stack size error.)


\begin{document}

\title{Syntactic Annotation by Interactive Proof Search}
\author[1]{Konstantin Sokolov}
\author[2]{Dimitri Timofeev}
\author[3]{Yury Kizhaev}
\affil[1]{Department of Mathematical Linguistics, St.~Petersburg State University, St.~Petersburg, Russia}
\affil[2]{Institute of Computer Science and Technology, Peter the Great St.~Petersburg Polytechnic University, St.~Petersburg, Russia}
\affil[3]{JetBrains Research, St.~Petersburg, Russia}
\date{}

\maketitle

% 1. Introduction
% CCG -- положительные аспекты формализма (выразительность -- mildly context sensitive, дистантные зависимости, свободный порядок слов, синт.-семант. интерфейс как для логической семантики, так и для дистрибутивной), подходит для моделирования синтаксиса русского языка; лексикализованный -- возможность обучания wide-coverage парсера по размеченному корпусу; мультимодальные расширения (как для и. Ламбека, так и CCG); отрицательные аспекты -- spurious ambiguity
% Подходы к получению корпуса с синт. разметкой -- а) полностью ручная, б) конвертация имеющегося корпуса другого формата (подход применен Hockenmeier, CCGBank, в) использование парсера, обучающегося по размечаемому корпусу (bootstrapping). 
% Особенности применения CCG в русском языке: отечественная традиция ориентирована на грамматики зависимостей (ср. сравнение парсеров на Диалог-2012); готового общедоступного корпуса корпуса для трансляции нет либо лиценционные сложности (SynTagRus); положительные аспекты -- традиция исследований в области категориальных грамматик (исчисление Ламбека, Пентус)
% Наша цель -- построить корпус русского языка с разметкой в формализме, относящемся к категориальным грамматикам (аналогично CCGBank), построить семантический парсер  (следуя [Curran, Clark] и работам по синт-сем интерфейсу для CCG (Clark, Bos, Grefenstette, Sadrzadeh и др.) Мы предлагаем подход на основе интерактивного поиска доказательств (proof assistants) с опорой на логическое направление в категориальных грамматиках, следуя работам французской школы (Moot, Retore, Cocuand, Huet, Girard). Это должно позволить значительно ускорить разметку за счет автоматизации рутинных операций и обеспечить качество.

% 2. Related Work
% Логический подход к синтаксису: a) parsing as deduction (Pereira, Warren) и др., b) исчисление Ламбека, с) комбинаторная логика для КГ -- CCG
% теоретико-типовая семантика -- использование формальных систем (Ранта и пр.)
% про proof assistants (исчисление индуктивных конструкций, зависимые типы, тактики и т.п.)
% Proof assistants for natural language syntax: а) Grail3 \parencite{moot2002proof}, б) библиотека Icharate для Coq \parencite{anoun2007approche}, в) CatLog -- парсер и прувер на языке Prolog в рамках Type-Logical Grammar, \parencite{morrill2012catlog}, г) серия работ по Natural Language Inference (NLI) в Coq в рамках теоретико-типового подхода к семантике, \parencite{chatzikyriakidis2014natural}
% про OpenCorpora и их tagset

% 3. Our approach
% Initial mapping of existing morphological tags to syntactic categories; 
% layered approach (multimodal grammars) -- use the mininimal subset where possible; manual intervention when the inference process is stuck; мера сложности разбора (какие правила задействованы в выводе, насколько сложные синт. категории использованы)
% добавление синт. категории для какой-то словоформы формирует конечный набор возможных разборов для тех предложений, в которых эта словоформа встречается; процесс построения вариантов разборов для этих предложений может быть полностью автоматическим, от пользователя требуется только аудит. В некоторых случаях даже выбор из возможных вариантов разборов можно автоматизировать (эвристика -- более простой разбор предпочтителен, сложность разбора считается автоматически)
% т.о. предполагаемая процедура разметки: предварительный этап -- начальный набор синт. категорий формируется вручную на основе морфологической информации, с его помощью размечается первоначальный набор предложений (максимально простые разборы); далее итеративно -- а) выбирается частично разобранное предложение (для которого не хватает одной словоформы), строится разбор (интерактивно, т.е. с подсказками от системы),  синт. категория для словоформы добавляется в БД, б) система строит автоматические разборы с использованием добавленной словоформы, ранжирует варианты по сложности, пользователь проводит аудит и подтвержает приемлемые разборы; на новый цикл

% 4. Discussion
% Оценки числа синт. категорий в получаемом подкорпусе (ср. CCGBank -- ок. 1200), субкатегоризация
% оценки объема наборов предложений под автоматическую разметку (очевидно, влияет частотность словоформ, но также и сложность предложений)
% оценки числа вариантов разборов для отдельных предложений (реальная омонимия, ложная омонимия). Но нас интересует получение одного канонического разбора в тексте, а не все варианты, поэтому для корпуса это не проблема (может быть проблемой для парсера)
% тактики доказательств -- какие способы разметки можно автоматизировать скриптованием (ср. тактики в Coq)
% имеется ли проблема циклических зависимостей? (Гипотетическая проблема: две словоформы, принять решение о синт. категории одной из них можно только зная синт. категорию другой и наоборот)
% crowdsourcing в нашем подходе не применим, но насколько высоки требования к теоретической подготовке аннотаторов? (Предположительно, меньшие, чем в традиционной ручной разметке, т.к. здесь количество сложных решений при разметке, требующих хорошей теоретической подготовке в области ситнаксиса, может быть сильно меньше)
% оценки скорости разметки, т.е. сколько всего времени потребуется на разметку OpenCorpora (ориентировочная цель -- разметка миллионного корпуса за год)
% возможность непосредственного переноса технологии на другие корпуса с морфологической разметкой и другие языки.

% 5. Conclusion
 

\section{Introduction}

\section{Ssdfdfs}

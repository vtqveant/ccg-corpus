\documentclass[a4paper]{report}

\usepackage[12pt]{extsizes} % report document class ignores the setting in \documentclass

\usepackage{afterpage}

%%% Кодировки и шрифты %%%
\usepackage[T2A]{fontenc}
\usepackage[utf8]{inputenc}
\usepackage[english,russian]{babel}

\usepackage{amssymb,amsfonts,amsmath,mathtext,amscd}

\usepackage{cmap}   % чтобы работал поиск по pdf
\usepackage{graphicx}
\pdfcompresslevel=9

%% красная строка
\usepackage{indentfirst}
\setlength{\parindent}{5.5ex}   % 5 символов

%% интерлиньяж
\linespread{1.1}                   

%%% Общее форматирование
\usepackage[singlelinecheck=off,center]{caption} % Многострочные подписи
\usepackage{soul} % Поддержка переносоустойчивых подчёркиваний и зачёркиваний

%%% Гиперссылки %%%
\usepackage[plainpages=false,pdfpagelabels=false]{hyperref}

\usepackage{geometry}
\geometry{a4paper,top=2cm,bottom=2cm,left=2.5cm,right=1cm}

%%% Выравнивание и переносы %%%
\sloppy
\clubpenalty=10000
\widowpenalty=10000

%%% оформление заголовков глав, секций и подсекций
\usepackage{titlesec}

\makeatletter
    \renewcommand{\l@section}{\@dottedtocline{1}{0.85cm}{0.85cm}}
    \renewcommand{\thesection}{\arabic{chapter}.\arabic{section}}
    \renewcommand{\section}{\@startsection{section}{1}{5.5ex}{-3.5ex plus -1ex minus -.2ex}{2.3ex plus.2ex}{\raggedright\hyphenpenalty=10000\normalfont\bfseries}}

    \renewcommand{\l@subsection}{\@dottedtocline{2}{1.25cm}{1.25cm}}
    \renewcommand{\thesubsection}{\arabic{chapter}.\arabic{section}.\arabic{subsection}}
    \renewcommand{\subsection}{\@startsection{subsection}{2}{5.5ex}{-3.5ex plus -1ex minus -.2ex}{2.3ex plus.2ex}{\raggedright\hyphenpenalty=10000\normalfont\bfseries}}
\makeatother

% \titleformat{command}[shape]{format}{label}{separation}{before-code}[after-code]
\titleformat{\section}[block]{\centering\hyphenpenalty=10000\large\bfseries}{\thesection.}{0pt}{\large} 
\titlespacing*{\section}{0pt}{-20pt}{30pt}
\titleformat{\chapter}[block]{\centering\hyphenpenalty=10000\normalfont\large\MakeUppercase\bfseries}{\thechapter. }{0pt}{\large\MakeUppercase}
\titlespacing*{\chapter}{0pt}{-20pt}{30pt}

%%% biblatex
\usepackage{csquotes} % recommended by pdflatex
\usepackage[%
    style = gost-authoryear-min,
    singletitle = false,
    autolang = other,
    language = auto,
    backend = biber,
    defernumbers = true,
    sortlocale = ru
]{biblatex}
\addbibresource{bibliography/bibliography.bib}

\defbibenvironment{bibliography}
{\enumerate{}
{\setlength{\leftmargin}{\bibhang}%
\setlength{\itemindent}{-\leftmargin}%
\setlength{\itemsep}{\bibitemsep}%
\setlength{\parsep}{\bibparsep}}}
{\endenumerate}
{\item}


\usepackage{hyperref}
\hypersetup{%
    colorlinks = true,
    hidelinks
}

%% graphics
\graphicspath{{images/}}

%% Gentzen style natural deduction proof trees
\usepackage{bussproofs}
\usepackage{latexsym}


%%% linguistic packages
\usepackage{drs}

%%% attribute-value matrices
\usepackage{avm}
\avmoptions{sorted,active}
\avmfont{\small\sc}
\avmvalfont{\small\rm}
\avmsortfont{\scriptsize\it}

\usepackage{tikz-qtree,tikz-qtree-compat}  % regular trees (e.g. GB style)
\usepackage{tikz-dependency}   % dependency trees (bracket style)
\usetikzlibrary{matrix,arrows}  % for commutative diagrams

\usepackage{gb4e}  % numbered lists for linguistic examples (IMPORTANT: If you use gb4e package, let it be the last \usepackage call in the document's preamble. Otherwise you may get exceeded parameter stack size error.)

\title{\vfill Синтаксическая разметка в формализме CCG\\ для проекта <<Открытый корпус>>}
\author{К.~В.~Соколов, {\small \textit{СПбГУ}}}
\date{\vfill Санкт-Петербург\\ \the\year}

\begin{document}

\maketitle
\tableofcontents    

\chapter{Введение}

\section{Синтаксический разбор как логический вывод}

Начала подхода заложены Ламбеком в работе \parencite{lambek1958mathematics}, термин введен в \parencite{pereira1983parsing}. В работе \parencite{konig1989parsing} рассматривается система натуральной дедукции для исчисления Ламбека без произведений. Также \parencite{moot2012logic}.

Задачи: алгоритм синтаксического анализа; изучение свойств формализма: разрешимость, вычислительная сложность, неоднозначность, корректность и полнота, существование нормальной формы и эффективной процедуры нормализации и пр.

Исчисление: формальный язык, множество аксиом (возможно, пустое), правила вывода, определение корректного вывода.

\section{Синтаксический разбор контекстно-свободных грамматик как вывод в системе гильбертовского типа}

Предложен в работе \parencite{pereira1983parsing}. В системе гильбертовского типа используется понятие линейного вывода.

Пример линейного вывода для логики высказываний.
Пример синтаксического разбора по \parencite{kallmeyer2010parsing}

\section{Исчисление Ламбека как секвенциальная система}

Предложена в работе \parencite{lambek1958mathematics}, см. тж. \parencite{moot2012logic}.

Правила вывода:


\begin{prooftree}
  \AxiomC{$\Gamma, B, \Gamma' \vdash C$}
  \AxiomC{$\Delta \vdash A$}
  \RightLabel{{\small $\textbackslash_h$}}
  \BinaryInfC{$\Gamma, A, A \textbackslash B, \Gamma' \vdash C$}
\end{prooftree}

\begin{prooftree}
  \AxiomC{$A, \Gamma \vdash C$}
  \RightLabel{{\small $\textbackslash_i, \Gamma \neq \epsilon$}}
  \UnaryInfC{$\Gamma \vdash A \textbackslash C$}
\end{prooftree}

\begin{prooftree}
  \AxiomC{$\Gamma, B, \Gamma' \vdash C$}
  \AxiomC{$\Delta \vdash A$}
  \RightLabel{{\small $/_h$}}
  \BinaryInfC{$\Gamma, B/A, \Delta, \Gamma' \vdash C$}
\end{prooftree}

\begin{prooftree}
  \AxiomC{$\Gamma, A \vdash C$}
  \RightLabel{{\small $/_i, \Gamma \neq \epsilon$}}
  \UnaryInfC{$\Gamma \vdash C/A$}
\end{prooftree}

\begin{prooftree}
  \AxiomC{$\Gamma, A, B, \Gamma' \vdash C$}
  \RightLabel{{\small $\bullet_h$}}
  \UnaryInfC{$\Gamma, A \bullet B, \Gamma' \vdash C$}
\end{prooftree}

\begin{prooftree}
  \AxiomC{$\Delta \vdash A$}
  \AxiomC{$\Gamma \vdash B$}
  \RightLabel{{\small $\bullet_i$}}
  \BinaryInfC{$\Delta, \Gamma \vdash A \bullet B$}
\end{prooftree}

\begin{prooftree}
  \AxiomC{$\Gamma \vdash A$}
  \AxiomC{$\Delta_1, A, \Delta_2 \vdash B$}
  \RightLabel{{\small \textit{сечение}}}
  \BinaryInfC{$\Delta_1, \Gamma, \Delta_2 \vdash B$}
\end{prooftree}

\begin{prooftree}
  \AxiomC{}
  \RightLabel{{\small \textit{аксиома}}}
  \UnaryInfC{$A \vdash A$}
\end{prooftree}

Замечание: \parencite{moot2012logic} пишут $B \textbackslash A$ там, где (Steedman et al.) пишут $A \textbackslash B$.



\nocite{*}
\printbibliography[resetnumbers=true]

\end{document}
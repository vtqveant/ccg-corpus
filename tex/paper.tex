\documentclass[a4paper,12pt]{article}

%%% Поля и разметка страницы %%%
\usepackage{lscape} % Для включения альбомных страниц
\usepackage{afterpage}
\usepackage{geometry} % Для последующего задания полей

%%% Кодировки и шрифты %%%
\usepackage[T2A]{fontenc}
\usepackage[utf8]{inputenc}
\usepackage[russian,english]{babel}
\usepackage{amssymb,amsfonts,amsmath,mathtext}
\usepackage{enumerate,float}

%% Gentzen style natural deduction proof trees
\usepackage{bussproofs}
\usepackage{latexsym}

%%% Оформление абзацев %%%
\usepackage{indentfirst} % Красная строка

%%% Общее форматирование
\usepackage[singlelinecheck=off,center]{caption} % Многострочные подписи
\usepackage{soul} % Поддержка переносоустойчивых подчёркиваний и зачёркиваний

%%% Гиперссылки %%%
\usepackage[plainpages=false,pdfpagelabels=false]{hyperref}

%%% Макет страницы %%%
\geometry{a4paper,top=2cm,bottom=2cm,left=2.5cm,right=2cm}

%%% Выравнивание и переносы %%%
\sloppy
\clubpenalty=10000
\widowpenalty=10000

\usepackage{cite}
\usepackage{hyperref}
\hypersetup{%
    colorlinks = true
}

\title{CCG Corpus}
\author{К.~В.~Соколов}
\date{}

\begin{document}

\maketitle

\section*{Синтаксический разбор как логический вывод}

Начала подхода заложены Ламбеком в работе \cite{lambek1958mathematics}, термин введен в \cite{pereira1983parsing}. В работе \cite{konig1989parsing} рассматривается система натуральной дедукции для исчисления Ламбека без произведений. Также \cite{moot2012logic}.

Задачи: алгоритм синтаксического анализа; изучение свойств формализма: разрешимость, вычислительная сложность, неоднозначность, корректность и полнота, существование нормальной формы и эффективной процедуры нормализации и пр.

Исчисление: формальный язык, множество аксиом (возможно, пустое), правила вывода, определение корректного вывода.

\section*{Синтаксический разбор контекстно-свободных грамматик как вывод в системе гильбертовского типа}

Предложен в работе \cite{pereira1983parsing}. В системе гильбертовского типа используется понятие линейного вывода.

Пример линейного вывода для логики высказываний.
Пример синтаксического разбора по \cite{kallmeyer2010parsing}

\section*{Исчисление Ламбека как секвенциальная система}

Предложена в работе \cite{lambek1958mathematics}, см. тж. \cite{moot2012logic}.

Правила вывода:


\begin{prooftree}
  \AxiomC{$\Gamma, B, \Gamma' \vdash C$}
  \AxiomC{$\Delta \vdash A$}
  \RightLabel{{\small $\textbackslash_h$}}
  \BinaryInfC{$\Gamma, A, A \textbackslash B, \Gamma' \vdash C$}
\end{prooftree}

\begin{prooftree}
  \AxiomC{$A, \Gamma \vdash C$}
  \RightLabel{{\small $\textbackslash_i, \Gamma \neq \epsilon$}}
  \UnaryInfC{$\Gamma \vdash A \textbackslash C$}
\end{prooftree}

\begin{prooftree}
  \AxiomC{$\Gamma, B, \Gamma' \vdash C$}
  \AxiomC{$\Delta \vdash A$}
  \RightLabel{{\small $/_h$}}
  \BinaryInfC{$\Gamma, B/A, \Delta, \Gamma' \vdash C$}
\end{prooftree}

\begin{prooftree}
  \AxiomC{$\Gamma, A \vdash C$}
  \RightLabel{{\small $/_i, \Gamma \neq \epsilon$}}
  \UnaryInfC{$\Gamma \vdash C/A$}
\end{prooftree}

\begin{prooftree}
  \AxiomC{$\Gamma, A, B, \Gamma' \vdash C$}
  \RightLabel{{\small $\bullet_h$}}
  \UnaryInfC{$\Gamma, A \bullet B, \Gamma' \vdash C$}
\end{prooftree}

\begin{prooftree}
  \AxiomC{$\Delta \vdash A$}
  \AxiomC{$\Gamma \vdash B$}
  \RightLabel{{\small $\bullet_i$}}
  \BinaryInfC{$\Delta, \Gamma \vdash A \bullet B$}
\end{prooftree}

\begin{prooftree}
  \AxiomC{$\Gamma \vdash A$}
  \AxiomC{$\Delta_1, A, \Delta_2 \vdash B$}
  \RightLabel{{\small \textit{сечение}}}
  \BinaryInfC{$\Delta_1, \Gamma, \Delta_2 \vdash B$}
\end{prooftree}

\begin{prooftree}
  \AxiomC{}
  \RightLabel{{\small \textit{аксиома}}}
  \UnaryInfC{$A \vdash A$}
\end{prooftree}

Замечание: \cite{moot2012logic} пишут $B \textbackslash A$ там, где (Steedman et al.) пишут $A \textbackslash B$.

\iffalse
%% дальше куски из Харпера

Центральное понятие логики -- понятие \textit{следования}, которое записывается $P_1, \dots, P_n \vdash P$ и выражает выводимость $P$ из $P_1,\dots, P_n$. Такая запись говорит о том, что $P$ выводимо по правилам логики, если $P_i$ даны в качестве аксиом. Следование обладает по меньшей мере двумя ключевыми структурными свойствами, позволяющими рассматривать его в качестве отношения предпорядка:

\begin{prooftree}
  \AxiomC{}
  \UnaryInfC{$P \vdash P$}
\end{prooftree}

\begin{prooftree}
  \AxiomC{$P \vdash Q$}
  \AxiomC{$Q \vdash R$}
  \BinaryInfC{$P \vdash R$}
\end{prooftree}

Кроме того, часто вводятся следующие дополнительные структурные свойства:

\begin{prooftree}
  \AxiomC{$P_1, \dots, P_n \vdash Q$}
  \UnaryInfC{$P_1, \dots, P_n, P_{n+1} \vdash Q$}
\end{prooftree}

\begin{prooftree}
  \AxiomC{$P_1,\dots,P_i,P_{i+1},\dots,P_n\vdash Q$}
  \UnaryInfC{$P_1,\dots,P_{i+1},P_{i},\dots,P_n\vdash Q$}
\end{prooftree}

\begin{prooftree}
  \AxiomC{$P_1,\dots,P_i,P_i,\dots,P_n\vdash Q$}
  \UnaryInfC{$P_1,\dots,P_i,\dots,P_n\vdash Q$}
\end{prooftree}

Они говорят о том, что ``дополнительные'' аксиомы не влияют на выводимость, ``переупорядочивание'' аксиом не играет роли, ``удвоение'' аксиом не играет роли. Эти условия кажутся неизбежными, но в т.\,н. субструктурных логиках любые из этих аксиом могут не иметь места.

В теории языков программирования базовая концепция -- \textit{суждение о типизации}, записыващееся как $x_1{:}A_1, \dots, x_n{:} A_n \vdash M{:}A$ и утверждающее, что $M$ есть выражение типа $A$, содержащее переменные $x_i$ типа $A_i$. Суждение о типизации должно удовлетворять следующим основным структурным свойствам:

\begin{prooftree}
  \AxiomC{}
  \UnaryInfC{$x:A\vdash x:A$}
\end{prooftree}

\begin{prooftree}
  \AxiomC{$y:B\vdash N:C \quad x:A\vdash M:B$}
  \UnaryInfC{$x:A\vdash [M/y]N:C$}
\end{prooftree}

Можно думать о переменных как об именах ``библиотек'', при этом первое свойство говорит о том, что использоваться может любая библиотека, а второе -- о замкнутости относительно компоновки (как в программе \textit{ld} в Unix и аналогичных), где $[M/x]N$ есть результат компоновки $x$ с библиотекой $M$ в выражении $N$. Обычно мы ожидаем, что аналоги аксиом ``дополнения'', ``переупорядочивания'' и ``удвоения'' будут выполняться и здесь, хотя это не обязательно. Их формулировка оставляется читателю в качестве упражнения.

В теории категорий основная концепция -- концепция \textit{отображения} $f:X \rightarrow Y$ между структурами $X$ и $Y$. Простейшими примерами являются, вероятно, множества и функции между ними, однако чаще рассматриваются, например, топологические пространства и непрерывные отображения между ними. Отображение удовлетворяет аналогичным структурным свойствам:

\begin{prooftree}
  \AxiomC{}
  \UnaryInfC{$\textit{id}_X : X \rightarrow X$}
\end{prooftree}

\begin{prooftree}
  \AxiomC{$f:X \rightarrow Y$}
  \AxiomC{$g:Y \rightarrow Z$}
  \BinaryInfC{$g \circ f : X \rightarrow Z$}
\end{prooftree}

\fi


\nocite{*}
\bibliographystyle{plain}
\bibliography{bibliography}

\end{document}